\section{Realisierung}

Die Implementierung erfolgt iterativ. D.h. wir entwickeln unterschiedliche ggf. aufeinander aufbauende Lösungen
und versuchen uns mit jedem Versuch zu verbessern. Unseren Fortschritt messen wir, indem wir in jeder
Iteration ein einreichbares Ergebnis erzeugen. Die Bewertung erfolgt somit über einen internen Algorithmus
von Kaggle.

Allgemein:
- Erst lokal testen mit kleinen Daten
- Dann auf gesamten Daten in Hive

\subsection{Implementierung}

\subsubsection{Iteration 0}

Als Erstes wollen wir eine Grundlage zur Bewertung folgender Versuche schaffen und das Einreichen einer
Lösung bei Kaggle üben. Dazu erzeugen wir eine Datei, die jedem Kunden die Kaufwahrscheinlichkeit 0\%
zuordnet. Hierzu verwenden wir eine rudimentäre Hive-Abfrage.

\begin{lstlisting}[language=SQL]
SELECT DISTINCT(h.id), 0.0 AS repeatProbability 
FROM test_history h
\end{lstlisting}

Die Bewertung von Kaggle sieht wie folgt aus:

\begin{tabular}{|c|c|}
	\hline \textbf{Platzierung} & \textbf{Bewertung} \\ 
	\hline 932 & 0.50000  \\ 
	\hline 
\end{tabular}

\subsubsection{Iteration 1}

Im nächsten Schritt wollen wir den ersten "`Prior Category Benchmark"' von Kaggle implementieren.
Dieser Benchmark ordnet Kunden, die bereits ein Produkt der Angebotskategorie gekauft haben, die
Wiederkaufwahrscheinlichkeit 1 zu. Allen anderen Kunden wird die Wahrscheinlichkeit 0 zugeordnet.

Wir bestimmen die Kunden mit einer Wahrscheinlichkeit von 1 über folgende Hive-Abfrage:
\begin{lstlisting}[language=SQL]
SELECT DISTINCT h.id, 1.0 AS repeatProbability
FROM test_history h INNER JOIN offers o ON (h.offer = o.offer)
INNER JOIN transactions t ON (t.id = h.id)
WHERE t.category = o.category
\end{lstlisting}

Um das Ergebnis bei Kaggle einzureichen, fehlen noch die Kunden mit einer Wahrscheinlichkeit von 0.
Um diese hinzuzufügen wurde ein Python-Skript entwickelt, dass eine unvollständige Einreichung
um die fehlenden Einträge erweitert. Für diese Kunden wird eine Wahrscheinlichkeit von 0 eingetragen.
Dieses Skript (s. Anhang \ref{code:complete-submission}) kann in weiteren Iterationen verwendet werden,
um sicher zu stellen, dass das Ergebnis vollständig ist und die Überprüfung von Kaggle übersteht. 

Die Bewertung von Kaggle ergibt wie erwartet:

\begin{tabular}{|c|c|}
	\hline \textbf{Platzierung} & \textbf{Bewertung} \\ 
	\hline 747 & 0.52000  \\ 
	\hline 
\end{tabular}

\subsubsection{Iteration 2}

In Iteration 1 haben wir den Benchmark mit der niedrigsten Bewertung implementiert. Jetzt wollen wir 
den besten Benchmark umsetzen. Der "`Prior (Brand \& Company \& Category) Benchmark"' ist sehr ähnlich
zu dem ersten Benchmark. Wir ordnen jedem Kunden, der schon einmal ein Produkt der Angebotsmarke,
des Angebotsunternehmens und der Angebotskategorie gekauft hat, eine 1 zu. Den anderen Kunden wird
eine Wahrscheinlichkeit von 0 zugeordnet.

\begin{lstlisting}[language=SQL]
SELECT DISTINCT h.id, 1.0 AS repeatProbability
FROM test_history h INNER JOIN offers o ON (h.offer = o.offer)
INNER JOIN transactions t ON (t.id = h.id)
WHERE t.category = o.category
  AND t.company = o.company
  AND t.brand = o.brand
\end{lstlisting}

Bevor wir das Ergebnis einreichen, verwenden wir das Skript zum Vervollständigen aus Iteration 1.

\begin{tabular}{|c|c|}
	\hline \textbf{Platzierung} & \textbf{Bewertung} \\ 
	\hline 627 & 0.56425  \\ 
	\hline 
\end{tabular}

\subsubsection{Iteration 3}

Nachdem wir in den vorherigen Iterationen die vorgegeben Benchmarks von Kaggle implementiert haben,
wollen wir jetzt ein richtiges Data-Mining-Verfahren verwenden. Dazu ermitteln wir im ersten Schritt
auf Basis der Transaktionsdaten Features für jeden Kunden. Bei der Wahl der Features orientieren
wir uns an den Ergebnissen von einem Kaggle-Nutzer (\url{http://mlwave.com/predicting-repeat-buyers-vowpal-wabbit/}).

Folgende Merkmale werden jeweils für die Marke (brand), das Unternehmen (company) und die Kategorie (category)
des Angebots (offer) ermittelt:
\begin{itemize}
	\item Für wie viele Produkte mit welchem Preis gilt das Angebot?
	\item Wie oft hat der Kunde in den letzen 30, 60, 90 und 180 Tagen gekauft? (Transaktionszahl)
	\item Wie viel hat der Kunde in den letzen 30, 60, 90 und 180 Tagen gekauft? (Anzahl gekaufter Produkte)
	\item Für wie viel Geld hat der Kunde in den letzen 30, 60, 90 und 180 Tagen gekauft? (Kosten)
\end{itemize}

Die Features werden über eine Reihe von Hive-Abfragen erstellt. Um die Zwischenergebnisse auch für
weitere Iterationen nutzen zu können, legen wir Tabellen mit den Ergebnissen an. Das erlaubt zusätzlich
einen strukturierten Aufbau der Merkmale.

Da die Kunden-Daten verteilt über die \texttt{train\_history} und die \texttt{test\_history} sind, 
bilden wir zuerst die Vereinigung der Kunden-Daten:
\begin{lstlisting}[language=SQL]
CREATE TABLE customers AS 
(
  SELECT te.id, te.offer, te.offerdate
    FROM test_history te
  UNION
  SELECT tr.id, tr.offer, tr.offerdate
    FROM train_history tr 
)
\end{lstlisting}
Danach ermitteln wir die Transaktionsdaten, die sich jeweils auf die Marke, das Unternehmen oder 
die Kategorie des Angebots für den Kunden beziehen. Dabei ermitteln wir zusätzlich, wie viele
Tage vor dem Angebot die Transaktion durchgeführt wurde. Beispielhaft ist hier die Abfrage
für die Marke dargestellt. Die anderen Anfragen werden analog erstellt und sind im Anhang
\ref{sql:filtered_tx} hinterlegt.
\begin{lstlisting}[language=SQL]
CREATE TABLE filtered_tx_brand AS
  SELECT t.id, t.purchasequantity, t.purchaseamount, (c.offerdate - t.date) AS daysbefore 
  FROM transactions t 
  INNER JOIN customers c ON (t.id = c.id)
  INNER JOIN offers o ON (c.offer = o.offer)
  WHERE o.brand = t.brand
\end{lstlisting}

Anschließend werden die gefilterten Transaktionsdaten gruppiert nach
Zeitintervallen ausgewertet. Die vollständigen Abfragen befinden sich
in Anhang \ref{sql:features_bcc}. Hier wird ein Auszug aus der
Abfrage für die Marken-Features dargestellt:
\begin{lstlisting}[language=SQL]
CREATE TABLE category_features AS
SELECT ever.id, ever.tx_count, ever.quantity, ever.cost,
before30.tx_count AS tx_count_30, before30.quantity AS quantity_30, before30.cost AS cost_30,
before60.tx_count AS tx_count_60, before60.quantity AS quantity_60, before60.cost AS cost_60,
before90.tx_count AS tx_count_90, before90.quantity AS quantity_90, before90.cost AS cost_90,
before180.tx_count AS tx_count_180, before180.quantity AS quantity_180, before180.cost AS cost_180
FROM 
(SELECT id, COUNT(*) AS tx_count, SUM(purchasequantity) AS quantity, 
SUM(purchaseamount) AS cost
FROM filtered_tx_category
GROUP BY id
) ever
LEFT JOIN
(SELECT id, COUNT(*) AS tx_count, SUM(purchasequantity) AS quantity, 
SUM(purchaseamount) AS cost
FROM filtered_tx_category
WHERE daysbefore <= 30
GROUP BY id
) before30
ON (ever.id = before30.id)
LEFT JOIN
...
\end{lstlisting}

Jetzt müssen die unterschiedlichen Features nur noch zusammengeführt werden. Dies passiert
in Anhang \ref{sql:features_combined_3} und umfasst einen Join zwischen den
unterschiedlichen Features auf Basis der Kunden-ID.

Um das Data-Mining-Tool Vorpal Wabbit verwenden zu können, müssen die Features in ein
passendes Format gewandelt werden. Das Ergebnis der Abfrage für die Features kann
in Hive als CSV abgespeichert werden. Die Umwandlung in das Format von Vorpal Wabbit
erfolgt über ein Python-Skript (s. Anhang \ref{code:features-to-vw}).

Anschließend erstellen wir ein Model auf Basis der Trainingsdaten und nutzen dieses
Model zur Voraussage der Wiederkaufwahrscheinlichkeiten für die Kunden aus den Testdaten.
\begin{lstlisting}
vw train.vw -c -k --passes 40 -l 0.85 -f model.vw --loss_function quantile --quantile_tau 0.6
vw test.vw -t -i model.vw -p shop.preds.txt
\end{lstlisting}
Das Ergebnis von Vorpal Wabbit muss jetzt nur noch in das Format zur Einreichung bei 
Kaggle konvertiert werden. Dies geschieht über das Python-Skript im Anhang 
\ref{code:vw-preds-to-csv}. Im Ergebnis können wir uns mit dieser Einreichung
schon etwas von den zuvor durchgeführten Benchmarks absetzen.

\begin{tabular}{|c|c|}
	\hline \textbf{Platzierung} & \textbf{Bewertung} \\ 
	\hline 499 & 0.58051  \\ 
	\hline 
\end{tabular}

\subsubsection{Iteration 4}

In dieser Iteration wird das Ergebnis aus Iteration 3 geringfügig verbessert.
Bei den Features, die wir bisher betrachtet haben, gehen wir von einer
positiven Wirkung auf das Kaufverhalten aus. Wir ergänzen diese Features
jetzt um negative Features.

Bei der Umwandlung der Features von CSV in das Format von Vorpal Wabbit 
erzeugen wir Features für Kunden, die noch nie Produkte der Marke,
des Unternehmens bzw. der Kategorie ihres Angebots gekauft haben.
\begin{lstlisting}[language=Python]
# Negative Features erzeugen, wenn nichts gekauft wurde
if len(row['b_tx_count']) == 0:
	output += ' b_never:1'
if len(row['co_tx_count']) == 0:
	output += ' co_never:1'
if len(row['ca_tx_count']) == 0:
	output += ' ca_never:1'
\end{lstlisting}

Bei Kaggle erzielen wir so ein Ergebnis, das bereits über dem des letzten Jahrgang liegt (Platz 452, Score 0.58519):

\begin{tabular}{|c|c|}
	\hline \textbf{Platzierung} & \textbf{Bewertung} \\ 
	\hline 344 & 0.58786  \\ 
	\hline 
\end{tabular}

\subsubsection{Iteration 5}

Jetzt erweitern wir die Features um Kombinationen. Wir führen neue Features ein, die den Wert 1 haben,
falls der Kunde von Marke und Unternehmen, Marke und Kategorie oder allen drei Eigenschaften bereits
gekauft hat.

\begin{lstlisting}[language=SQL]
SELECT f.*, 
  CAST(b_tx_count > 0 AND co_tx_count > 0 AND ca_tx_count > 0 AS INTEGER) AS any_b_co_ca,
  CAST(b_tx_count > 0 AND ca_tx_count > 0 AS INTEGER) AS any_b_ca,
  CAST(b_tx_count > 0 AND co_tx_count > 0 AS INTEGER) AS any_b_co
FROM test_features f

SELECT f.*, 
  CAST(b_tx_count > 0 AND co_tx_count > 0 AND ca_tx_count > 0 AS INTEGER) AS any_b_co_ca,
  CAST(b_tx_count > 0 AND ca_tx_count > 0 AS INTEGER) AS any_b_ca,
  CAST(b_tx_count > 0 AND co_tx_count > 0 AS INTEGER) AS any_b_co
FROM train_features f
\end{lstlisting}

Wir verwenden wie zuvor Vorpal Wabbit, um die eine Vorhersage zu generieren:

\begin{tabular}{|c|c|}
	\hline \textbf{Platzierung} & \textbf{Bewertung} \\ 
	\hline 241 & 0.59355  \\ 
	\hline 
\end{tabular}

\subsubsection{Iteration 6}
Bisher haben wir nur einen kleinen Teil der Daten zu Features verarbeitet, da wir nur Transaktionen
beachtet haben, die sich auf das entsprechende Angebot für den Kunden beziehen. Jetzt führen wir noch
die globalen Features "`Anzahl der Transaktionen"' und "`Gesamte Umsatzsumme"' ein. Diese Features
erzeugen wir in eignen Tabelle und joinen diese zu den bereits generierten Features (s. Anhang \ref{code:totals}).

\begin{lstlisting}[language=SQL]
CREATE TABLE totals AS
  SELECT t.id, COUNT(*) AS total_count, SUM(t.purchaseamount) AS total_spent
  FROM transactions t
  GROUP BY t.id
\end{lstlisting}

Das Ergebnis fällt enttäuschend aus und liegt leicht unter der bisherigen Bestmarke:
\begin{tabular}{|c|c|}
	\hline \textbf{Platzierung} & \textbf{Bewertung} \\ 
	\hline 262 & 0.59228  \\ 
	\hline 
\end{tabular}
	
\subsection{Amazon Web Services}

\subsubsection{Tabellenanlage}


\begin{lstlisting}[style=hive]
CREATE EXTERNAL TABLE offers (offer bigint, category bigint, quantity bigint, company bigint, offervalue double, brand bigint)
row format delimited fields terminated by ','
lines terminated by '\n'
STORED as TEXTFILE
LOCATION 's3://de.whs.fdt.kaggle.acquire-valued-shoppers- challenge/input_data_orig/offers/';
\end{lstlisting}

\subsubsection{Anfragen ausführen}

Anhang mit Skripten

