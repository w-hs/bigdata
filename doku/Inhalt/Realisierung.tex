\section{Implementierung}

Die Implementierung erfolgt iterativ. D.h. wir entwickeln unterschiedliche ggf. aufeinander aufbauende Lösungen
und versuchen uns mit jedem Versuch zu verbessern. Unseren Fortschritt messen wir, indem wir in jeder
Iteration ein einreichbares Ergebnis erzeugen. Die Bewertung erfolgt somit über einen internen Algorithmus
von Kaggle.

\subsection{Iteration 0}

Als Erstes wollen wir eine Grundlage zur Bewertung folgender Versuche schaffen und das Einreichen einer
Lösung bei Kaggle üben. Dazu erzeugen wir eine Datei, die jedem Kunden die Kaufwahrscheinlichkeit 0\%
zuordnet. Hierzu verwenden wir eine rudimentäre Hive-Abfrage.

\begin{lstlisting}[language=SQL]
SELECT DISTINCT(h.id) AS id, 
       0.0            AS repeatProbability 
FROM test_history h
\end{lstlisting}

Die Bewertung von Kaggle sieht wie folgt aus:

\begin{tabular}{|c|c|}
	\hline \textbf{Platzierung} & \textbf{Bewertung} \\ 
	\hline 932 & 0.50000  \\ 
	\hline 
\end{tabular}

\subsection{Iteration 1}
\subsection{Amazon Web Services}
\subsection{Iteration 2}
