\section{Hive-Abfragen}

\subsection{Kunden-Daten}
\begin{lstlisting}[language=SQL]
CREATE TABLE customers AS 
(
  SELECT te.id, te.offer, te.offerdate
    FROM test_history te
  UNION
  SELECT tr.id, tr.offer, tr.offerdate
    FROM train_history tr 
)
\end{lstlisting}

\subsection{Transaktionen gefiltert nach Marke, Unternehmen oder Kategorie}
\label{sql:filtered_tx}
\begin{lstlisting}[language=SQL]
-- Gefiltert nach Marke (brand)
CREATE TABLE filtered_tx_brand AS
  SELECT t.id, t.purchasequantity, t.purchaseamount, 
         (c.offerdate - t.date) AS daysbefore 
  FROM filtered_transactions t 
  INNER JOIN customers c ON (t.id = c.id)
  INNER JOIN offers o ON (c.offer = o.offer)
  WHERE o.brand = t.brand
  
-- Gefiltert nach Unternehmen (company)
CREATE TABLE filtered_tx_company AS
  SELECT t.id, t.purchasequantity, t.purchaseamount, 
         (h.offerdate - t.date) AS daysbefore 
  FROM transactions t 
  INNER JOIN test_history h ON (t.id = h.id)
  INNER JOIN offers o ON (h.offer = o.offer)
  WHERE o.company = t.company
  
-- Gefiltert nach Kategorie (category)
CREATE TABLE filtered_tx_category AS
  SELECT t.id, t.purchasequantity, t.purchaseamount, 
         (c.offerdate - t.date) AS daysbefore 
  FROM transactions t 
  INNER JOIN customers c ON (t.id = c.id)
  INNER JOIN offers o ON (c.offer = o.offer)
  WHERE o.category = t.category
\end{lstlisting}

\subsection{Features bezogen auf Marke, Unternehmen oder Kategorie}
\label{sql:features_bcc}
\begin{lstlisting}[language=SQL]
CREATE TABLE category_features AS
SELECT ever.id, ever.tx_count, ever.quantity, ever.cost,
before30.tx_count AS tx_count_30, before30.quantity AS quantity_30, before30.cost AS cost_30,
before60.tx_count AS tx_count_60, before60.quantity AS quantity_60, before60.cost AS cost_60,
before90.tx_count AS tx_count_90, before90.quantity AS quantity_90, before90.cost AS cost_90,
before180.tx_count AS tx_count_180, before180.quantity AS quantity_180, before180.cost AS cost_180
FROM 
(SELECT id, COUNT(*) AS tx_count, SUM(purchasequantity) AS quantity, 
 SUM(purchaseamount) AS cost
 FROM filtered_tx_category
 GROUP BY id
) ever
LEFT JOIN
(SELECT id, COUNT(*) AS tx_count, SUM(purchasequantity) AS quantity, 
 SUM(purchaseamount) AS cost
 FROM filtered_tx_category
 WHERE daysbefore <= 30
 GROUP BY id
) before30
ON (ever.id = before30.id)
LEFT JOIN
(SELECT id, COUNT(*) AS tx_count, SUM(purchasequantity) AS quantity, 
 SUM(purchaseamount) AS cost
 FROM filtered_tx_category
 WHERE daysbefore <= 60
 GROUP BY id
) before60
ON (ever.id = before60.id)
LEFT JOIN
(SELECT id, COUNT(*) AS tx_count, SUM(purchasequantity) AS quantity, 
 SUM(purchaseamount) AS cost
 FROM filtered_tx_category
 WHERE daysbefore <= 90
 GROUP BY id
) before90
ON (ever.id = before90.id)
LEFT JOIN 
(SELECT id, COUNT(*) AS tx_count, SUM(purchasequantity) AS quantity, 
 SUM(purchaseamount) AS cost
 FROM filtered_tx_category
 WHERE daysbefore <= 180
 GROUP BY id
) before180
ON (ever.id = before180.id)
\end{lstlisting}

\subsection{Features kombinieren (Iteration 3)}
\label{sql:features_combined_3}
\begin{lstlisting}[language=SQL]
-- Features der Kunden, für die Vorhersagen gemacht werden sollen
CREATE TABLE test_features AS
  SELECT 
    h.id, 'u' AS repeater,
    o.quantity AS o_quantity, o.offervalue,

    b.tx_count AS b_tx_count, b.quantity AS b_quantity, b.cost AS b_cost,
    b.tx_count_30 AS b_tx_count_30, b.quantity_30 AS b_quantity_30, b.cost_30 AS b_cost_30,
    b.tx_count_60 AS b_tx_count_60, b.quantity_60 AS b_quantity_60, b.cost_60 AS b_cost_60,
    b.tx_count_90 AS b_tx_count_90, b.quantity_90 AS b_quantity_90, b.cost_90 AS b_cost_90,
    b.tx_count_180 AS b_tx_count_180, b.quantity_180 AS b_quantity_180, b.cost_180 AS b_cost_180,

    co.tx_count AS co_tx_count, co.quantity AS co_quantity, co.cost AS co_cost,
    co.tx_count_30 AS co_tx_count_30, co.quantity_30 AS co_quantity_30, co.cost_30 AS co_cost_30,
    co.tx_count_60 AS co_tx_count_60, co.quantity_60 AS co_quantity_60, co.cost_60 AS co_cost_60,
    co.tx_count_90 AS co_tx_count_90, co.quantity_90 AS co_quantity_90, co.cost_90 AS co_cost_90,
    co.tx_count_180 AS co_tx_count_180, co.quantity_180 AS co_quantity_180, co.cost_180 AS co_cost_180,

    ca.tx_count AS ca_tx_count, ca.quantity AS ca_quantity, ca.cost AS ca_cost,
    ca.tx_count_30 AS ca_tx_count_30, ca.quantity_30 AS ca_quantity_30, ca.cost_30 AS ca_cost_30,
    ca.tx_count_60 AS ca_tx_count_60, ca.quantity_60 AS ca_quantity_60, ca.cost_60 AS ca_cost_60,
    ca.tx_count_90 AS ca_tx_count_90, ca.quantity_90 AS ca_quantity_90, ca.cost_90 AS ca_cost_90,
    ca.tx_count_180 AS ca_tx_count_180, ca.quantity_180 AS ca_quantity_180, ca.cost_180 AS ca_cost_180

  FROM test_history h
  INNER JOIN offers o ON (h.offer = o.offer)
  LEFT JOIN brand_features b ON (h.id = b.id)
  LEFT JOIN company_features co ON (h.id = co.id)
  LEFT JOIN category_features ca ON (h.id = ca.id)

-- Features der Kunden, für die Wiederkaufsdaten bekannt sind
CREATE TABLE train_features AS
  SELECT 
    h.id, h.repeater,
    o.quantity AS o_quantity, o.offervalue,

    b.tx_count AS b_tx_count, b.quantity AS b_quantity, b.cost AS b_cost,
    b.tx_count_30 AS b_tx_count_30, b.quantity_30 AS b_quantity_30, b.cost_30 AS b_cost_30,
    b.tx_count_60 AS b_tx_count_60, b.quantity_60 AS b_quantity_60, b.cost_60 AS b_cost_60,
    b.tx_count_90 AS b_tx_count_90, b.quantity_90 AS b_quantity_90, b.cost_90 AS b_cost_90,
    b.tx_count_180 AS b_tx_count_180, b.quantity_180 AS b_quantity_180, b.cost_180 AS b_cost_180,

    co.tx_count AS co_tx_count, co.quantity AS co_quantity, co.cost AS co_cost,
    co.tx_count_30 AS co_tx_count_30, co.quantity_30 AS co_quantity_30, co.cost_30 AS co_cost_30,
    co.tx_count_60 AS co_tx_count_60, co.quantity_60 AS co_quantity_60, co.cost_60 AS co_cost_60,
    co.tx_count_90 AS co_tx_count_90, co.quantity_90 AS co_quantity_90, co.cost_90 AS co_cost_90,
    co.tx_count_180 AS co_tx_count_180, co.quantity_180 AS co_quantity_180, co.cost_180 AS co_cost_180,

    ca.tx_count AS ca_tx_count, ca.quantity AS ca_quantity, ca.cost AS ca_cost,
    ca.tx_count_30 AS ca_tx_count_30, ca.quantity_30 AS ca_quantity_30, ca.cost_30 AS ca_cost_30,
    ca.tx_count_60 AS ca_tx_count_60, ca.quantity_60 AS ca_quantity_60, ca.cost_60 AS ca_cost_60,
    ca.tx_count_90 AS ca_tx_count_90, ca.quantity_90 AS ca_quantity_90, ca.cost_90 AS ca_cost_90,
    ca.tx_count_180 AS ca_tx_count_180, ca.quantity_180 AS ca_quantity_180, ca.cost_180 AS ca_cost_180

  FROM train_history h
  INNER JOIN offers o ON (h.offer = o.offer)
  LEFT JOIN brand_features b ON (h.id = b.id)
  LEFT JOIN company_features co ON (h.id = co.id)
  LEFT JOIN category_features ca ON (h.id = ca.id)
\end{lstlisting}

\subsection{Kombinierte Features hinzufügen (Iteration 5)}

\begin{lstlisting}[language=SQL]
-- Test-Features um ausgewählte kombinierte Features erweitern
SELECT f.*, 
  CAST(b_tx_count > 0 AND co_tx_count > 0 AND ca_tx_count > 0 AS INTEGER) AS any_b_co_ca,
  CAST(b_tx_count > 0 AND ca_tx_count > 0 AS INTEGER) AS any_b_ca,
  CAST(b_tx_count > 0 AND co_tx_count > 0 AS INTEGER) AS any_b_co
FROM test_features f

-- Train-Features um ausgewählte kombinierte Features erweitern
SELECT f.*, 
  CAST(b_tx_count > 0 AND co_tx_count > 0 AND ca_tx_count > 0 AS INTEGER) AS any_b_co_ca,
  CAST(b_tx_count > 0 AND ca_tx_count > 0 AS INTEGER) AS any_b_ca,
  CAST(b_tx_count > 0 AND co_tx_count > 0 AS INTEGER) AS any_b_co
FROM train_features f
\end{lstlisting}

\subsection{Globale Features (Iteration 6)}
\label{code:totals}
\begin{lstlisting}[language=SQL]
-- Erstellen der Tabelle mit den globalen Features
CREATE TABLE totals AS
  SELECT t.id, COUNT(*) AS total_count, SUM(t.purchaseamount) AS total_spent
  FROM transactions t
  GROUP BY t.id

-- Join der globlen Features zu den bereits bestehenden Test-Features
SELECT f.*, t.*
FROM test_features f
INNER JOIN totals t ON (f.id = t.id)

-- Join der globlen Features zu den bereits bestehenden Train-Features
SELECT f.*, t.*
FROM train_features f
INNER JOIN totals t ON (f.id = t.id)
\end{lstlisting}

\section{Python-Skripts}
\subsection{complete-submission.py}
\label{code:complete-submission}
\begin{lstlisting}[language=Python]
#!/usr/bin/env python

# Diese Skript vervollständigt eine Kaggle-Einreichung.
# Fehlenden Kunden wird eine Wahrscheinlichkeit von 0 
# zugeordnet.
# Die Datei sampleSubmissions.csv muss sich im aktuellen 
# Verzeichnis befinden

import sys

samplesFile = open('sampleSubmission.csv', 'r')
samplesFile.readline()
samples = { }
for line in samplesFile:
	id, prob = line.split(',', 1)
	samples[id] = prob

# Erste Zeile ignorieren
sys.stdin.readline()
# Und die geforderte Kopfzeile ausgeben
sys.stdout.write("id,repeatProbability\n")

for line in sys.stdin:
	id, prob = line.split(',', 1)
	if (id in samples):
		del samples[id]
	sys.stdout.write(id + ',' + prob)

	for id in samples:
		sys.stdout.write(id + ',' + samples[id])
\end{lstlisting}

\subsection{features-to-vw.py (Iteration 3)}
\label{code:features-to-vw}
\begin{lstlisting}[language=Python]
#!/usr/bin/env python

import sys
import csv

input = csv.DictReader(sys.stdin, delimiter=',')
for row in input:
	# Ausgabe beginnt mit der Klassifizierung:
	# hat gekauft (1), hat nicht gekauft (0) oder unbekannt (0.5)
	output = '0.5'
	if row['repeater'] == 't':
		output = '1'
	elif row['repeater'] == 'f':
		output = '0'

	# Customer ID eintragen und Feature-Liste beginnen
	output += ' \'' + row['id'] + ' |f'

	# Spalten in Features übersetzen
	for key in row:
		if key != 'id' and  key != 'repeater' and len(row[key]) > 0:
			output += ' ' + key + ':' + row[key]

	print(output)
\end{lstlisting}

\subsection{features-to-vw.py (Iteration 4)}
\label{code:features-to-vw-4}
\begin{lstlisting}[language=Python]
#!/usr/bin/env python

import sys
import csv

input = csv.DictReader(sys.stdin, delimiter=',')
for row in input:
    # Ausgabe beginnt mit der Klassifizierung:
    # hat gekauft (1), hat nicht gekauft (0) oder unbekannt (0.5)
    output = '0.5'
    if row['repeater'] == 't':
        output = '1'
    elif row['repeater'] == 'f':
        output = '0'

    # Customer ID eintragen und Feature-Liste beginnen
    output += ' \'' + row['id'] + ' |f'

    # Spalten in Features übersetzen
    for key in row:
    if key != 'id' and  key != 'repeater' and key != 'total_count' and key != 'total_spent' and len(row[key]) > 0:
        output += ' ' + key + ':' + row[key]

    # Negative Features erzeugen, wenn nichts gekauft wurde
    if len(row['b_tx_count']) == 0:
        output += ' b_never:1'
    if len(row['co_tx_count']) == 0:
        output += ' co_never:1'
    if len(row['ca_tx_count']) == 0:
        output += ' ca_never:1'

    print(output)
\end{lstlisting}

\subsection{vw-preds-to-csv.py}
\label{code:vw-preds-to-csv}
\begin{lstlisting}[language=Python]
#!/usr/bin/env python

import sys

# Das Ergebnis von Vorpal Wabbit gibt ein Paar aus 
# Wahrscheinlichkeit und Kunden-ID pro Zeile und durch ein
# Leerzeichen getrennt aus.
# Dieses Programm konviertiert diese Format in da von Kaggle
# geforderte CSV-Format.

print('id,repeatProbability')

for line in sys.stdin:
	prob, id = line.split(' ', 1)
	print(id[:-1] + ',' + prob)
\end{lstlisting}
	
	