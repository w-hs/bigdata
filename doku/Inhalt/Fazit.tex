\section{Schlussteil}

\subsection{Ausblick}

Es konnte bereits während des Projektes ein gutes Ergebnis erzielt
werden. Dennoch wurden aufgrund der knappen Zeit nicht alle Ideen
umgesetzt. Weitere Möglichkeiten sehen wir im Bereich der nicht
verwendeten Daten und der Nutzung eines Offline-Learning-Verfahrens.
Des Weiteren besteht die Möglichkeit, die verwendeten Features auf ihre
Aussagekraft beziehungsweise Qualität zu untersuchen.

\subsubsection{Verwendung aller Daten}
Es werden bisher hauptsächlich Transaktionen berücksichtigt, die in Beziehung zu Gutscheinen stehen.
Bei diesen Transaktionen muss die Marke, die Kategorie oder das Unternehmen mit dem entsprechenden Gutschein übereinstimmen.
Diese Daten könnten in einem nächsten Iterationsschritt ebenfalls verwertet werden, wozu zunächst die folgenden Ansätze zusammengetragen wurden:
	
\begin{itemize}
\item Für jeden Kunden soll der kompletten Umsatz pro Quartal beziehungsweise pro Jahr zu ermittelt werden. 
Die Idee ist, dass Leute die viel Geld umsetzen, weniger gut durch Gutscheine ansprechbar sind, als Leute die wenig zum Umsatz beitragen. 
 
\item Kunden, die in regelmässigen Abständen einkaufen, kommen sehr wahrscheinlich wieder.
Beispielsweise werden wöchentlich die Nahrungsvorräte auf dem Heimweg nach der Arbeit aufgefrischt.
Die Frequenz des Einkaufens kann über die Auswartung der Zeitangaben der Transaktionen ermittelt werden.

\item Clustering von oft zusammen gekauften Marken, Kategorien und Unternehmen.
Auf dieser Basis Features für die Kunden definieren.
	
\item Kunden die große Menge (purchasequantity) gekauft haben, kommen wahrscheinlich wieder. 
Sie benötigen keine Gutscheine, da der Preis so schon gut ist oder der Markt zum Einkauf gut gelegen ist. 
Realisierung: Was ist die Definition von viel? 
Erstellen eines Mittelwertes über die Spalte - jeder Wert der über dem Mittelwert liegt, gilt als viel.
\end{itemize}

\subsubsection{Bewertung von Features}	
Sowohl bei den verwendeten Features als auch bei den oben vorgestellten Ideen ist bisher unklar, inwiefern diese Features das Kaufverhalten positiv, negativ oder überhaupt beeinflussen.
Innerhalb der Vorträge ist zur Sprache gekommen, dass statistische Auswertungsverfahren zur Verfügung stehen, welche Aufschluss über die Qualität der Features treffen können. Da schlechte Features das Ergebnis negativ verfälschen können, ist die Ansatz besonders interessant.
Zur Umsetzung müsste die Korrelation zwischen den einzelnen Features
und dem Wiederkaufverhalten gemessen werden.
Dann kann ein Grenzwert festgelegt werden, ab dem ein Feature
zur Voraussage genutzt wird.

\subsubsection{Offline Learing Verfahren}	
	

\subsection{Fazit}
