\section{Einleitung}
Im Zeitalter von Google, Amazon und Facebook ist das Thema "`Big-Data"' allgegenwärtig. So auch im Modul Fortgeschrittene Datenbanktechniken im zweiten Semester des Masterstudiengangs Verteilte Systeme. Im Rahmen eines Projektes sollte das Thema unter einer beliebigen Aufgabenstellung selbstständig untersucht und erarbeitet werden. Dabei haben wir uns für das Projekt "`Shoppers-Challenge"' von Kaggle, das im nächsten Abschnitt genauer beschrieben wird, entschieden. Bei der Umsetzung haben wir basierend auf dem Hadoop-Framework verschiedene Technologien wie beispielsweise MapReduce-Algorithmen kennengelernt.

\subsection{Aufgabenstellung}
Ziel dieses Projekts ist es, ein möglichst hohes Rating bei der "`Shoppers Challange"' auf kaggle.com zu erreichen.

Die Plattform kaggle.com bietet Wettbewerbe im Bereich "`Big-data"' an, wobei hier größtenteils Vorhersagemodelle und Analysen erstellt werden sollen. Firmen und Wissenschaftler können auf dieser Plattform Aufgaben und zugehörige Daten zur Verfügung stellen, die von Statistikern und Data-Minern aus der ganzen Welt bearbeitet werden. Die Lösungen werden dabei nach ihrer Qualität, bezogen auf die Genauigkeit der Vorhersage, in einem Ranking aufgeführt und teilweise sogar prämiert. Damit bietet die Plattform kaggle.com für die Aufgabensteller den enormen Vorteil, dass sie aus vielen Strategien die beste auswählen können und die Bearbeiter gleichzeitig die Qualität ihrer Lösung mit denen der Konkurrenz vergleichen können.

Die "`Shoppers Challenge"' beschäftigt sich mit der Frage, ob ein Kunde der einen Gutschein genutzt hat, zu einem "`treuen"' Kunden wird und in Zukunft noch weitere Produkte kauft. Zur Lösung des Problems muss ein Vorhersagemodell erstellt werden, das mit Hilfe der vom Aufgabensteller bereitgestellten Daten, die Wahrscheinlichkeit für einen erneuten Kauf für jeden Kunden vorhersagt. 

\newpage
\subsection{Projektmanagement}
Unser Projekt unterteilt sich insgesamt in folgende drei Phasen:

\begin{figure}[H]
\centering
\includegraphics[width=0.8\linewidth]{Bilder/ProjektplanAllgemein}
\caption{Die drei Projektphasen}
\label{fig:ProjektplanAllgemein}
\end{figure}

Im Rahmen der Einarbeitungsphase soll zunächst eine Arbeitsumgebung in Form einer virtuellen Maschine (VM) ausgewählt werden. Nachdem wir uns für ein konkretes "`Big-Data-Projekt"' entschieden haben, wollen wir zunächst das Datenmodell skizzieren und uns Gedanken zu einem geeigneten Data-Mining-Verfahren machen. Basierend auf diesem Verfahren soll die Phase mit einer Technologieentscheidung, zur Umsetzung des Projekts, abgeschlossen werden.

\begin{figure}[h]
\centering
\includegraphics[width=1\linewidth]{Bilder/ProjektEinarbeitung}
\caption{Die Einarbeitungsphase}
\label{fig:ProjektEinarbeitung}
\end{figure}

Die Implementierungsphase soll iterativ gestaltet werden und mit einer initialen einfachen Umsetzung beginnen. Basierend auf der ersten Implementierung wollen wir uns stetig verbessern. Im Zuge der iterativen Verbesserungen soll der Umstieg auf die Amazon-Web-Services vollzogen werden. Zur Bewertung der Zwischenergebnisse beenden wir jede Iteration mit der Einreichung des aktuellen Zwischenergebnisses bei Kaggle. Mit Abgabe der finalen Implementierung wollen wir aus der Implementierungsphase in die Abschlussphase übergehen.

\begin{figure}[h]
\centering
\includegraphics[width=1\linewidth]{Bilder/ProjektImplementierung}
\caption{Die Implementierungsphase}
\label{fig:ProjektImplementierung}
\end{figure}

In dieser Phase reichen wir das finale Ergebnis ein. Anschließend wollen wir die Projektdokumentation fertigstellen und unsere Erkenntnisse im Rahmen einer Abschlusspräsentation aufbereiten. Nach der Benotung lassen wir das Projekt in gemütlicher Atmosphäre ausklingen.

\begin{figure}[h]
\centering
\includegraphics[width=1\linewidth]{Bilder/ProjektAbschluss}
\caption{Die Abschlussphase}
\label{fig:ProjektAbschluss}
\end{figure}


%- Phasen
%	- Einarbeitung
%	- Implementierung (iterativ)
%	- Projektabschluss
%	
%- Meilensteine
%	- Technologieentscheidung treffen
%	- "Erste Submission"
%	- Wechsel von Lokal VM nach AWS
%	- "Finale Submission" 
%	- Dokumentation abgeschlossen
%	- Präsentation abgeschlossen
%	
%	TODO: terminieren
%	
%- ggf. verwendete Werkzeuge (Git, Drive)





