\documentclass[
    12pt, % Schriftgröße
    DIV10,
    ngerman, % für Umlaute, Silbentrennung etc.
    a4paper, % Papierformat
    oneside, % einseitiges Dokument
    titlepage, % es wird eine Titelseite verwendet
    parskip=half, % Abstand zwischen Absätzen (halbe Zeile)
    headings=normal, % Größe der Überschriften verkleinern
    listof=totoc, % Verzeichnisse im Inhaltsverzeichnis aufführen
    bibliography=totoc, % Literaturverzeichnis im Inhaltsverzeichnis aufführen
    index=totoc, % Index im Inhaltsverzeichnis aufführen
    captions=tableheading, % Beschriftung von Tabellen unterhalb ausgeben
    final % Status des Dokuments (final/draft)
]{scrartcl}

% Pakete

% Schrift ----------------------------------------------------------------------
\usepackage{lmodern} % bessere Fonts

% Paket für Kopfzeilen und Fußzeilen
\usepackage[
    automark, % Kapitelangaben in Kopfzeile automatisch erstellen
    headsepline, % Trennlinie unter Kopfzeile
    ilines % Trennlinie linksbündig ausrichten
]{scrpage2}

% Anpassung an Landessprache
\usepackage[ngerman]{babel}

% Umlaute ----------------------------------------------------------------------
%   Umlaute/Sonderzeichen wie äöüß direkt im Quelltext verwenden (CodePage).
%   Erlaubt automatische Trennung von Worten mit Umlauten.
% ------------------------------------------------------------------------------
\usepackage[utf8]{inputenc}
\usepackage[T1]{fontenc}
\usepackage{textcomp} % Euro-Zeichen etc.

\usepackage[babel,german=quotes]{csquotes}

% Bei Änderungen müssen die *.aux und *.bbl Dateien manuell gelöscht werden
% Sonst kommt es zu einem Fehler bei der Erstellung
%\usepackage[round, sort, comma, numbers]{natbib}
%\usepackage[round, numbers]{natbib}
%\bibliographystyle{abbrvdin}
%

\usepackage{scrhack}

\usepackage[citestyle=alphabetic,bibstyle=alphabetic,backend=biber]{biblatex}
\bibliography{Bibliographie}
\DefineBibliographyStrings{german}{
  bibliography = {Literaturverzeichnis},
}
\setlength\bibitemsep{8pt}

\usepackage{xpatch}
\xpretobibmacro{author}{\mkbibbold\bgroup}{}{}
\xapptobibmacro{author}{\egroup}{}{}
\xpretobibmacro{editor}{\mkbibbold\bgroup}{}{}
\xapptobibmacro{editor}{\egroup}{}{}
\xpretobibmacro{editor+others}{\mkbibbold\bgroup}{}{}
\xapptobibmacro{editor+others}{\egroup}{}{}
\xpretobibmacro{bbx:editor}{\mkbibbold\bgroup}{}{}
\xapptobibmacro{bbx:editor}{\egroup}{}{}

% Grafiken ---------------------------------------------------------------------
% Einbinden von JPG-Grafiken ermöglichen
\usepackage[dvips,final]{graphicx}
\graphicspath{{Bilder/}}

% Paket zur Verwendung zusätzlicher Positionsbefehle
\usepackage{float}

% Befehle aus AMSTeX für mathematische Symbole z. B. \boldsymbol \mathbb --------
\usepackage{amsmath,amsfonts}

% Einfache Definition der Zeilenabstände und Seitenränder etc. -----------------
\usepackage{setspace}
\usepackage{geometry}

% zum Umfließen von Bildern ----------------------------------------------------
\usepackage{floatflt}

% Farbdefinitionen
\usepackage{xcolor} 
\definecolor{hellgelb}{rgb}{1,1,0.9}
\definecolor{colKeys}{rgb}{0,0,1}
\definecolor{colIdentifier}{rgb}{0,0,0}
\definecolor{colComments}{rgb}{1,0,0}
\definecolor{colString}{rgb}{0,0.5,0}
\definecolor{whgreen}{RGB}{113,177,41}
\definecolor{dvBlue}{RGB}{0,85,160}

% URL verlinken, lange URLs umbrechen etc. -------------------------------------
\usepackage{url}

% PDF-Optionen -----------------------------------------------------------------
\usepackage[
    bookmarks,
    bookmarksopen=true,
    colorlinks=true,
% diese Farbdefinitionen zeichnen Links im PDF farblich aus
    linkcolor=dvBlue, % einfache interne Verkn�pfungen
    anchorcolor=black,% Ankertext
    citecolor=dvBlue, % Verweise auf Literaturverzeichniseintr�ge im Text
    filecolor=magenta, % Verkn�pfungen, die lokale Dateien �ffnen
    menucolor=dvBlue, % Acrobat-Men�punkte
    urlcolor=dvBlue, 
% diese Farbdefinitionen sollten für den Druck verwendet werden (alles schwarz)
%    linkcolor=black, % einfache interne Verkn�pfungen
%    anchorcolor=black, % Ankertext
%    citecolor=black, % Verweise auf Literaturverzeichniseintr�ge im Text
%    filecolor=black, % Verkn�pfungen, die lokale Dateien �ffnen
%    menucolor=black, % Acrobat-Men�punkte
%    urlcolor=black, 
    %backref, % Inkompatibel mit BibLateX
    plainpages=false, % zur korrekten Erstellung der Bookmarks
    pdfpagelabels, % zur korrekten Erstellung der Bookmarks
    hypertexnames=false, % zur korrekten Erstellung der Bookmarks
    linktocpage % Seitenzahlen anstatt Text im Inhaltsverzeichnis verlinken
]{hyperref}

% fortlaufendes Durchnummerieren der Fußnoten ----------------------------------
\usepackage{chngcntr}
%\counterwithout{footnote}{chapter}

% Formatierung von Listen ändern -----------------------------------------------
\usepackage{paralist} % itemize, enumerate

% bei der Definition eigener Befehle benötigt
\usepackage{ifthen} % Vielleicht nicht nötig

% definiert u.a. die Befehle \ und \listoftodos
\usepackage{todonotes}
\reversemarginpar

% sorgt dafür, dass Leerzeichen hinter parameterlosen Makros nicht als Makroendezeichen interpretiert werden
\usepackage{xspace}

\usepackage{tabularx} % Tabellenspalten mit variabler Breite
\usepackage{wrapfig}  % Schriftumflossene Bilder


% Seitenstil

% Zeilenabstand 1,5 Zeilen
\onehalfspacing

% Seitenränder
% bottom is 20 + X, weil Footer nicht berücksichtigt wird
\geometry{paper=a4paper,left=30mm,right=20mm,top=20mm, bottom=38mm, footskip=8mm}

% Kopf- und Fußzeilen ----------------------------------------------------------
% Kopf- und Fußzeile auch auf Kapitelanfangsseiten
%\renewcommand*{\chapterpagestyle}{scrheadings} 
% Schriftform der Kopfzeile
%\renewcommand{\headfont}{\normalfont}

% Kopfzeile
\ihead{\headmark} % links
\chead{}
%\ohead{\includegraphics[scale=0.1]{Bilder/DJLogo.png}} % rechts
\setlength{\headheight}{20mm} % Höhe der Kopfzeile
% Kopfzeile über den Text hinaus verbreitern
%\setheadwidth[0pt]{textwithmarginpar} 
\setheadsepline[text]{0.4pt} % Trennlinie unter Kopfzeile

% Fußzeile
%\ifoot{} % links
\cfoot{} % mitte
\ofoot{\pagemark} % rechts
%\setfootsepline[text]{0.4pt} % Trennlinie unter Kopfzeile



% Schusterjungen und Hurenkinder vermeiden
%\clubpenalty = 10000
%\widowpenalty = 10000 
%\displaywidowpenalty = 10000

% Verringert den Abstand über den Überschriften
%\renewcommand*{\chapterheadstartvskip}{\vspace*{-\topskip}}

% Irgendwas mit Silbentrennung in MonoSpace (texttt)
\newcommand{\origttfamily}{}% sollte noch nicht definiert sein!
\let\origttfamily=\ttfamily % alte Definition von \ttfamily sichern
\renewcommand{\ttfamily}{\origttfamily \hyphenchar\font=`\-}


% Eigene Befehle und typographische Auszeichnungen für diese

\newcommand{\bs}{$\backslash$}

% einige Befehle zum Zitieren --------------------------------------------------
\newcommand{\Zitat}[2][\
empty]{\ifthenelse{\equal{#1}{\empty}}{\citep{#2}}{\citep[#1]{#2}}}

% zum Ausgeben von Autoren
\newcommand{\AutorName}[1]{\textsc{#1}}
\newcommand{\Autor}[1]{\AutorName{\citeauthor{#1}}}

% Produktnamen
\newcommand{\produkt}[1]{\textbf{#1}}

\newcommand{\code}[1]{\texttt{#1}}

% zum Einbinden von Programmcode -----------------------------------------------
\usepackage{listings}
\usepackage{xcolor}
\usepackage{textcomp}
\lstset{
    float=hbp,
    basicstyle=\ttfamily\color{black}\small,
    identifierstyle=\color{colIdentifier},
    keywordstyle=\color{colKeys},
    stringstyle=\color{colString},
    commentstyle=\color{colComments},
    %columns=flexible,
    tabsize=2,
    frame=single,
    extendedchars=true,
    showspaces=false,
    showstringspaces=false,
    numbers=left,
    numberstyle=\ttfamily\small,
    numbersep=5pt,
    breaklines=true,
    backgroundcolor=\color{hellgelb},
    breakautoindent=true
}

\lstset{literate=
  {á}{{\'a}}1 {é}{{\'e}}1 {í}{{\'i}}1 {ó}{{\'o}}1 {ú}{{\'u}}1
  {Á}{{\'A}}1 {É}{{\'E}}1 {Í}{{\'I}}1 {Ó}{{\'O}}1 {Ú}{{\'U}}1
  {à}{{\`a}}1 {è}{{\'e}}1 {ì}{{\`i}}1 {ò}{{\`o}}1 {ù}{{\`u}}1
  {À}{{\`A}}1 {È}{{\'E}}1 {Ì}{{\`I}}1 {Ò}{{\`O}}1 {Ù}{{\`U}}1
  {ä}{{\"a}}1 {ë}{{\"e}}1 {ï}{{\"i}}1 {ö}{{\"o}}1 {ü}{{\"u}}1
  {Ä}{{\"A}}1 {Ë}{{\"E}}1 {Ï}{{\"I}}1 {Ö}{{\"O}}1 {Ü}{{\"U}}1
  {â}{{\^a}}1 {ê}{{\^e}}1 {î}{{\^i}}1 {ô}{{\^o}}1 {û}{{\^u}}1
  {Â}{{\^A}}1 {Ê}{{\^E}}1 {Î}{{\^I}}1 {Ô}{{\^O}}1 {Û}{{\^U}}1
  {œ}{{\oe}}1 {Œ}{{\OE}}1 {æ}{{\ae}}1 {Æ}{{\AE}}1 {ß}{{\ss}}1 {Δ}{{$\Delta$}}1
  {ç}{{\c c}}1 {Ç}{{\c C}}1 {ø}{{\o}}1 {å}{{\r a}}1 {Å}{{\r A}}1
  {€}{{\EUR}}1 {£}{{\pounds}}1
}

\usepackage{microtype}

\renewcommand*{\dictumwidth}{.41\textwidth}
\renewcommand*{\dictumrule}{}
\renewcommand*{\dictumauthorformat}[1]{--- #1}
\setkomafont{dictumauthor}{%
\scshape
}

\definecolor{bluekeywords}{rgb}{0,0,1}
\definecolor{greencomments}{rgb}{0,0.5,0}
\definecolor{redstrings}{rgb}{0.64,0.08,0.08}
\definecolor{xmlcomments}{rgb}{0.5,0.5,0.5}
\definecolor{types}{rgb}{0.17,0.57,0.68}

\lstdefinestyle{csharp}
{
    language=[Sharp]C,
    captionpos=b,
    %numbers=left, %Nummerierung
    %numberstyle=\tiny, % kleine Zeilennummern
    showspaces=false,
    showtabs=false,
    breaklines=true,
    showstringspaces=false,
    breakatwhitespace=true,
    escapeinside={(*@}{@*)},
    commentstyle=\color{greencomments},
    morekeywords={partial, var, value, get, set},
    keywordstyle=\color{bluekeywords},
    stringstyle=\color{redstrings},
    basicstyle=\ttfamily\small,
}

\lstdefinestyle{hive}
{
    language=SQL,
    captionpos=b,
    %numbers=left, %Nummerierung
    %numberstyle=\tiny, % kleine Zeilennummern
    showspaces=false,
    showtabs=false,
    breaklines=true,
    showstringspaces=false,
    breakatwhitespace=true,
    escapeinside={(*@}{@*)},
    commentstyle=\color{greencomments},
    morekeywords={bigint, double, decimal, char, string, row, format, lines, location, DELIMITED, FIELDS, TERMINATED, STORED, TEXTFILE},
    keywordstyle=\color{bluekeywords},
    stringstyle=\color{redstrings},
    basicstyle=\ttfamily\small,
}

\lstdefinestyle{xml}
{
    language=XML,
    captionpos=b,
    %numbers=left, %Nummerierung
    %numberstyle=\tiny, % kleine Zeilennummern
    showspaces=false,
    showtabs=false,
    breaklines=true,
    showstringspaces=false,
    breakatwhitespace=true,
    escapeinside={(*@}{@*)},
    commentstyle=\color{greencomments},
    morekeywords={encoding},
    keywordstyle=\color{bluekeywords},
    stringstyle=\color{redstrings},
    basicstyle=\ttfamily\small,
}

\colorlet{punct}{red!60!black}
\definecolor{background}{HTML}{EEEEEE}
\definecolor{delim}{RGB}{20,105,176}
\colorlet{numb}{magenta!60!black}

\lstdefinelanguage{json}{
    basicstyle=\normalfont\ttfamily,
    numbers=left,
    numberstyle=\scriptsize,
    stepnumber=1,
    numbersep=8pt,
    showstringspaces=false,
    breaklines=true,
    frame=single,
    literate=
     *{0}{{{\color{numb}0}}}{1}
      {1}{{{\color{numb}1}}}{1}
      {2}{{{\color{numb}2}}}{1}
      {3}{{{\color{numb}3}}}{1}
      {4}{{{\color{numb}4}}}{1}
      {5}{{{\color{numb}5}}}{1}
      {6}{{{\color{numb}6}}}{1}
      {7}{{{\color{numb}7}}}{1}
      {8}{{{\color{numb}8}}}{1}
      {9}{{{\color{numb}9}}}{1}
      {:}{{{\color{punct}{:}}}}{1}
      {,}{{{\color{punct}{,}}}}{1}
      {\{}{{{\color{delim}{\{}}}}{1}
      {\}}{{{\color{delim}{\}}}}}{1}
      {[}{{{\color{delim}{[}}}}{1}
      {]}{{{\color{delim}{]}}}}{1},
}

%\usepackage{caption} 
%\captionsetup[table]{skip=100pt}